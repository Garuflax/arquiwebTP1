\documentclass{beamer}

%\usepackage[utf8]{inputenc} %Para acentos en UTF8 (Prueba: � � � � � � � � � � � �)

\usepackage[latin1]{inputenc}
\usepackage[spanish]{babel}
\usepackage[absolute,overlay]{textpos}
\setlength{\TPHorizModule}{1mm}
\setlength{\TPVertModule}{1mm}

\usetheme{Warsaw}

\usecolortheme[rgb={1,0.48,0.0}]{structure}%divido los RGB por 252
\setbeamercolor{block title}{fg=white,bg=verdeuca}
\xdefinecolor{verdeuca}{rgb}{0.0,0.48,0.54}
\xdefinecolor{naranjauca}{rgb}{1,0.48,0.0}
\setbeamercolor{palette quaternary}{fg=white,bg=verdeuca}

\setbeamertemplate{navigation symbols}{}

\usepackage{color}

\title{Presentaci�n Trabajo Pr�ctico 1}
\subtitle{Arquitecturas de Aplicaciones Web}

\author[David A. Gonz�lez M�rquez]{David Alejandro Gonz�lez M�rquez}
\institute{Departamento de Computaci�n\\
Facultad de Ciencias Exactas y Naturales\\
Universidad de Buenos Aires}
\date{18-11-20}

\definecolor{celeste}{rgb}{.255,.41,.884}
\definecolor{rojo}{rgb}{1, 0, 0}

\begin{document}

\frame[plain]{\titlepage}

\begin{frame}
    \frametitle{Agenda}
    \begin{itemize}\setlength\itemsep{2em}
     \item[-] Introducci�n a Bochs
     \item[-] Bootloader, Compilado y Enlazado
     \item[-] Pasaje a Modo Protegido
     \item[-] Presentaci�n del TP3
    \end{itemize}
\end{frame}

\begin{frame}
    \frametitle{�Porque usar bochs?}
    El procesador posee intrucciones que no se pueden usar a nivel de usuario. \\
    Por lo tanto, si queremos acceder a estos mecanismos debemos estar en lugar del sistema operativo.
    \vspace{0.3cm}
    \begin{itemize}
    \item[-] Utilizar instrucciones de nivel privilegiado
    \item[-] Acceder a los mecanismos de manejo de memoria
    \item[-] Cambiar modos del procesador
    \item[-] Controlar interrupciones
    \end{itemize}
\end{frame}

\begin{frame}
    \frametitle{VM con Debugger}
    Un debugger como \texttt{gdb} es un \textcolor{blue}{proceso} m�s.
    No puede monitorear el sistema operativo.
    %\begin{center}
    %\includegraphics[scale=0.5]{img/bochsdbg.png}\\
    %Bochs + Debugger
    %\end{center}
    Necesitamos un debugger \textcolor{blue}{en} la Virtual Machine.
\end{frame}


\begin{frame}[plain]
    \begin{center}
    \Huge !`!`!`Gracias!!! \\
    \vspace{1cm}
    \Huge �Preguntas?
    \end{center}
\end{frame}

\end{document}
